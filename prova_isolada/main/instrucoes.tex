%%%%%%%%%%%%%%%%%%%%%%%%%%%%%%%%%%%%%%%%%%%%%%%%%%%%%%%%%%%%%%%%%%%%%%%%%%%%%%%%
%%% Folha de instruções padronizadas da UVV para a prova
\pdfbookmark[1]{Instruções}{instrucoes}
\begin{figure}[H]
\begin{center}
\includegraphics[scale=0.3]{imagens/cover/logo-uvv.png}
\end{center}	
\end{figure}

\vspace{-1cm}

\begin{table}[H]
\centering
\begin{tabular}{|llll|}
\hline
\multicolumn{2}{|l|}{\textbf{Disciplina:} Introdução à Computação} & \multicolumn{1}{l|}{\multirow{3}{*}{\textbf{\makecell{Nota:\\ \,\\ \,}}}} & \multirow{3}{*}{\textbf{\makecell{Coordenador:\\ \,\\ \,}}} \\ \cline{1-2}
\multicolumn{2}{|l|}{\textbf{Professor:} Abrantes Araújo Silva Filho\hspace{4.72cm}\,}                 & \multicolumn{1}{l|}{}                                &                                                   \\ \cline{1-2}
\multicolumn{2}{|l|}{\textbf{Aluno:}}                                                 & \multicolumn{1}{l|}{}                                &                                                   \\ \hline
\multicolumn{1}{|l|}{\textbf{Turma:} \hspace{6.3cm}\,}    & \multicolumn{1}{l|}{\textbf{Semestre:}}    & \multicolumn{2}{l|}{\textbf{Valor:} 10 pontos}                                                           \\ \hline
\multicolumn{1}{|l|}{\textbf{Data:}}     & \multicolumn{3}{l|}{\textbf{Avaliação:} 1ª Avaliação} \\ \hline
\end{tabular}
\end{table}

\begin{center}
\fbox{\setlength\fboxsep{0.3cm}\fbox{\parbox{15.4cm}{
\begin{center}
\textbf{INSTRUÇÕES PARA A REALIZAÇÃO DESTA AVALIAÇÃO:}
\end{center}
\begin{itemize}[leftmargin=*]
\item Esta é a primeira avaliação da disciplina Introdução à Computação,
      e refere-se ao conteúdo dos capítulos 1, 2, 5 e 6 do livro texto
      de referência da disciplina. O objetivo desta prova é verificar seu
      aprendizado em relação aos conceitos iniciais e fundamentais da
      computação.
\item Esta avaliação contém 60 questões objetivas, \textbf{todas obrigatórias}.
\item \textbf{Leia atentamente cada questão!} As questões objetivas terão uma,
      e apenas uma única, resposta a ser marcada.
\item \textbf{Desligue o celular} antes de começar e coloque o celular sobre a
      mesa do professor. Ao final da prova você poderá pegar seu celular.
\item A avaliação é \textbf{individual} e \textbf{sem consulta}.
\item As questões podem ser respondidas, na prova, com lápis ou caneta. Ao final
      da prova \textbf{\underline{VOCÊ DEVERÁ PREENCHER O GABARITO COM CANETA
      DE COR PRETA}} \textbf{\underline{OU AZUL}} para que seja feita a correção
      automatizada através da leitura do padrão de respostas marcado no
      gabarito.
\item \textbf{CUIDADO ao preencher o gabarito} pois eles são \textbf{INDIVIDUAIS
      e NOMEADOS} (cada estudante tem um gabarito próprio específico, com um
      código de identificação único). \textbf{Questões rasuradas são anuladas}
      pelo software de leitura do gabarito.
\item Ao final da prova, \textbf{DEVOLVA A PROVA E O GABARITO} para o professor.
\item Siga todas as normas de \textbf{Integridade Acadêmica} da disciplina.
      Alunos que forem flagrados com qualquer espécie de ``cola'' ou trocando
      informações com outros alunos terão suas avaliações recolhidas, as notas
      zeradas, e a situação será encaminhada para a coordenação para a aplicação
      das penalidades previstas pela universidade.
\item Com 90 minutos de avaliação você terá 1,5\,min por questão, em média, mas
      lembre-se do tempo de preenchimento do gabarito.
\item Boa avaliação!
\end{itemize}
%\vspace{1cm}
}}}
\end{center}
