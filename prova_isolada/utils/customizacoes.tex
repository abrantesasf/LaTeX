%%%%%%%%%%%%%%%%%%%%%%%%%%%%%%%%%%%%%%%%%%%%%%%%%%%%%%%%%%%%%%%%%%%%%%%%%%%%%%%%
% customizacoes.tex
%
% Arquivo de configuração de packages para uso o LaTeX, conforme minhas
% preferências e modelos pessoais.
%
% Para maiores informações, visite:
%    https://github.com/abrantesasf/latex
%
% NÃO ALTERE SE NÃO SOUBER O QUE ESTÁ FAZENDO!
%%%%%%%%%%%%%%%%%%%%%%%%%%%%%%%%%%%%%%%%%%%%%%%%%%%%%%%%%%%%%%%%%%%%%%%%%%%%%%%%


%%%%%%%%%%%%%%%%%%%%%%%%%%%%%%%%%%%%%%%%%%%%%%%%%%%%%%%%%%%%%%%%%%%%%%%%%%%%%%%%
%%% Vários comandinhos úteis e outras gambiarras

% Commando para ``italizar´´ palavras em inglês (e outras línguas!):
\newcommand{\ingles}[1]{\textit{#1}}

% Comando para escrever uma função e símbolos em fonte monoespaçada:
\newcommand{\funcao}[1]{\textbf{\texttt{#1}}}
\newcommand{\funcaob}[1]{\fbox{\texttt{#1}}}
\newcommand{\simbolo}[1]{\texttt{#1}}

% Commando para colocar o espaço correto entre um número e sua unidade:
\newcommand{\unidade}[2]{\ensuremath{#1\,\mathrm{#2}}}
\newcommand{\unidado}[2]{{#1}\,{#2}}

% Produz ordinal masculino ou feminino dependendo do segundo argumento:
\newcommand{\ordinal}[2]{%
#1%
\ifthenelse{\equal{a}{#2}}%
{\textordfeminine}%
{\textordmasculine}}

% Comando para colocar o autor da citação corretamente justificado à direita:
% Modificado de: https://latex.org/forum/viewtopic.php?p=15605&sid=216895679326b531e85caec779e1d710#p15605
%\newcommand*{\fontequote}[2]{%
%   \unskip\hspace*{1em plus 1fill}%
%   \linebreak\hspace*{\fill}\mbox{#1}\vspace{-0.15cm}
%   \linebreak\hspace*{\fill}\mbox{#2}
%}

% Comandos para os quadros padronizados dos livros de matemática:
\newcommand{\abrevideo}[1]{%
\begin{tcolorbox}[colback=yellow!5!white,colframe=yellow!75!black,
       title=\textbf{Vídeo(s) de referência: #1}]}
\newcommand{\fechavideo}{\end{tcolorbox}}

% Comandos para os ícones comuns:
\newcommand{\youtube}{\faYoutube}

% Comandos para álgebra relacional (testados em XeLaTeX):
\def\ojoin{\setbox0=\hbox{$\Join$}%
\rule[0.02ex]{.4em}{.5pt}\llap{\rule[0.95ex]{.4em}{.5pt}}}
\def\leftouterjoin{\mathbin{\ojoin\mkern-7.8mu\Join}}
\def\rightouterjoin{\mathbin{\Join\mkern-7.8mu\ojoin}}
\def\fullouterjoin{\mathbin{\ojoin\mkern-7.8mu\Join\mkern-7.8mu\ojoin}}

% Comando para números dentro de círculos
% https://tex.stackexchange.com/questions/7032/good-way-to-make-textcircled-numbers
\newcommand*\circled[1]{\tikz[baseline=(char.base)]{
            \node[shape=circle,draw,inner sep=2pt] (char) {#1};}}

% Comando para a linguagem Snap!
\newcommand{\snap}{Snap\textit{!}}

% Figura inline:
\newcommand{\inlinefigure}[1]{%
  \begingroup\normalfont
  \includegraphics[height=\fontcharht\font`\B]{#1}%
  \endgroup
}

% Para o estudo da glibc:
\newcommand{\glibc}{\texttt{glibc}}
