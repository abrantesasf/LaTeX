%%%%%%%%%%%%%%%%%%%%%%%%%%%%%%%%%%%%%%%%%%%%%%%%%%%%%%%%%%%%%%%%%%%%%%%%%%%%%%%%
% hp12c.tex
%
% Arquivo de configurações para o livro sobre a calculadora HP-12C.
%
% Para maiores informações, visite:
%    https://github.com/abrantesasf/hp12c_fundamentos
%
% NÃO ALTERE SE NÃO SOUBER O QUE ESTÁ FAZENDO!
%%%%%%%%%%%%%%%%%%%%%%%%%%%%%%%%%%%%%%%%%%%%%%%%%%%%%%%%%%%%%%%%%%%%%%%%%%%%%%%%


%%%%%%%%%%%%%%%%%%%%%%%%%%%%%%%%%%%%%%%%%%%%%%%%%%%%%%%%%%%%%%%%%%%%%%%%%%%%%%%%
%%% Configura cores importantes
\definecolor{stackRegister}{RGB}{221,242,232}
%\definecolor{stackContent}{RGB}{235,245,250}
\definecolor{stackContent}{RGB}{251,252,253}
\definecolor{display}{RGB}{150,255,212}
\definecolor{comentarios}{RGB}{195,255,255}


%%%%%%%%%%%%%%%%%%%%%%%%%%%%%%%%%%%%%%%%%%%%%%%%%%%%%%%%%%%%%%%%%%%%%%%%%%%%%%%%
%%% Teclas da HP-12C:
\tikzstyle{abstract}=[rectangle, draw=black, rounded corners, fill=gray!20,
                      drop shadow, text centered, text=black, text width=8mm]
\tikzstyle{fkey}=[rectangle, draw=black, rounded corners, fill=orange,
                  drop shadow, text centered,  text=black, text width=8mm]
\tikzstyle{gkey}=[rectangle, draw=black, rounded corners, fill=blue!40,
                  drop shadow, text centered,  text=black, text width=8mm]

\newcommand{\mykey}[2]{%
\begin{tikzpicture} \node (Item) [abstract, rectangle split, rectangle split parts=2]    
{\textbf{\scriptsize{#1}} \nodepart{second}\textbf{\tiny{#2}}};%
\end{tikzpicture}}

\newcommand{\myfkey}{%
\begin{tikzpicture} \node (Item) [fkey, rectangle split, rectangle split parts=2]    
{\textbf{\footnotesize{f}} \nodepart{second}};%
\end{tikzpicture}}

\newcommand{\mygkey}{%
\begin{tikzpicture} \node (Item) [gkey, rectangle split, rectangle split parts=2]  
{\textbf{\footnotesize{g}} \nodepart{second}};%
\end{tikzpicture}}

\NewDocumentCommand{\tecla}{oommm}
 {% #1 = width (optional)
  % #2 = inner alignment (optional)
  % #3 = frame color
  % #4 = background color
  % #5 = text
  \IfValueTF{#1}
   {\IfValueTF{#2}
    {\fcolorbox{#3}{#4}{\makebox[#1][#2]{#5}}}
    {\fcolorbox{#3}{#4}{\makebox[#1]{#5}}}%
   }
   {\fcolorbox{#3}{#4}{#5}}%
}

% Primeira linha da HP-12C
\newcommand{\teclaamort}{\tecla{black}{orange!40}{AMORT}}
\newcommand{\teclaint}{\tecla{black}{orange!40}{INT}}
\newcommand{\teclanpv}{\tecla{black}{orange!40}{NPV}}
\newcommand{\teclarnd}{\tecla{black}{orange!40}{RND}}
\newcommand{\teclairr}{\tecla{black}{orange!40}{IRR}}
\newcommand{\teclan}{\tecla{black}{white}{n}}
\newcommand{\teclai}{\tecla{black}{white}{i}}
\newcommand{\teclapv}{\tecla{black}{white}{PV}}
\newcommand{\teclapmt}{\tecla{black}{white}{PMT}}
\newcommand{\teclafv}{\tecla{black}{white}{FV}}
\newcommand{\teclachs}{\tecla{black}{white}{CHS}}
\newcommand{\teclasete}{\tecla{black}{white}{7}}
\newcommand{\teclaoito}{\tecla{black}{white}{8}}
\newcommand{\teclanove}{\tecla{black}{white}{9}}
\newcommand{\tecladiv}{\tecla{black}{white}{$\div$}}
\newcommand{\tecladozex}{\tecla{black}{blue!20}{12\,$\times$}}
\newcommand{\tecladozediv}{\tecla{black}{blue!20}{12\,$\div$}}
\newcommand{\teclacfzero}{\tecla{black}{blue!20}{CF$_0$}}
\newcommand{\teclacfj}{\tecla{black}{blue!20}{CF$_j$}}
\newcommand{\teclanj}{\tecla{black}{blue!20}{N$_j$}}
\newcommand{\tecladate}{\tecla{black}{blue!20}{DATE}}
\newcommand{\teclabeg}{\tecla{black}{blue!20}{BEG}}
\newcommand{\teclaend}{\tecla{black}{blue!20}{END}}
\newcommand{\teclamem}{\tecla{black}{blue!20}{MEM}}

% Segunda linha da HP-12C
\newcommand{\teclabond}{\tecla{white}{white}{\color{orange}{\textbf{BOND}}}}
\newcommand{\teclaprice}{\tecla{black}{orange!40}{PRICE}}
\newcommand{\teclaytm}{\tecla{black}{orange!40}{YTM}}
\newcommand{\tecladepreciation}{\tecla{white}{white}{\color{orange}{\textbf{DEPRECIATION}}}}
\newcommand{\teclasl}{\tecla{black}{orange!40}{SL}}
\newcommand{\teclasoyd}{\tecla{black}{orange!40}{SOYD}}
\newcommand{\tecladb}{\tecla{black}{orange!40}{DB}}
\newcommand{\teclaelevado}{\tecla{black}{white}{$y^x$}}
\newcommand{\teclainverso}{\tecla{black}{white}{$1/x$}}
\newcommand{\teclaporcentot}{\tecla{black}{white}{\%\,T}}
\newcommand{\tecladeltaporcento}{\tecla{black}{white}{$\Delta$\,\%}}
\newcommand{\teclaporcento}{\tecla{black}{white}{\%}}
\newcommand{\teclaeex}{\tecla{black}{white}{EEX}}
\newcommand{\teclaquatro}{\tecla{black}{white}{4}}
\newcommand{\teclacinco}{\tecla{black}{white}{5}}
\newcommand{\teclaseis}{\tecla{black}{white}{6}}
\newcommand{\teclavezes}{\tecla{black}{white}{$\times$}}
\newcommand{\teclaraiz}{\tecla{black}{blue!20}{$\sqrt{x}$}}
\newcommand{\teclaeelevado}{\tecla{black}{blue!20}{$e^x$}}
\newcommand{\teclaln}{\tecla{black}{blue!20}{LN}}
\newcommand{\teclafrac}{\tecla{black}{blue!20}{FRAC}}
\newcommand{\teclaintg}{\tecla{black}{blue!20}{INTG}}
\newcommand{\tecladeltadias}{\tecla{black}{blue!20}{$\Delta$\,DYS}}
\newcommand{\tecladmy}{\tecla{black}{blue!20}{D.MY}}
\newcommand{\teclamdy}{\tecla{black}{blue!20}{M.DY}}
\newcommand{\teclaxw}{\tecla{black}{blue!20}{$\overline{x}\,w$}}

% Terceira linha da HP-12C (incluindo enter, exceto lstx)
\newcommand{\teclaclear}{\tecla{white}{white}{\color{orange}{\textbf{CLEAR}}}}
\newcommand{\teclapr}{\tecla{black}{orange!40}{P/R}}
\newcommand{\teclasoma}{\tecla{black}{orange!40}{$\sum$}}
\newcommand{\teclaprgm}{\tecla{black}{orange!40}{PRGM}}
\newcommand{\teclafin}{\tecla{black}{orange!40}{FIN}}
\newcommand{\teclareg}{\tecla{black}{orange!40}{REG}}
\newcommand{\teclaprefix}{\tecla{black}{orange!40}{PREFIX}}
\newcommand{\teclars}{\tecla{black}{white}{R/S}}
\newcommand{\teclasst}{\tecla{black}{white}{SST}}
\newcommand{\teclar}{\tecla{black}{white}{R\,$\downarrow$}}
\newcommand{\teclatrocaxy}{\tecla{black}{white}{$x{\gtrless}y$}}
\newcommand{\teclaclx}{\tecla{black}{white}{CL\,$x$}}
\newcommand{\teclaenter}{\tecla{black}{white}{ENTER}}
\newcommand{\teclaum}{\tecla{black}{white}{1}}
\newcommand{\tecladois}{\tecla{black}{white}{2}}
\newcommand{\teclatres}{\tecla{black}{white}{3}}
\newcommand{\teclamenos}{\tecla{black}{white}{$-$}}
\newcommand{\teclapse}{\tecla{black}{blue!20}{PSE}}
\newcommand{\teclabst}{\tecla{black}{blue!20}{BST}}
\newcommand{\teclagto}{\tecla{black}{blue!20}{GTO}}
\newcommand{\teclaxmenorigualy}{\tecla{black}{blue!20}{$x \leqslant  y$}}
\newcommand{\teclaxigualzero}{\tecla{black}{blue!20}{$x = 0$}}
\newcommand{\teclaxr}{\tecla{black}{blue!20}{$\widehat{x}\,,r$}}
\newcommand{\teclayr}{\tecla{black}{blue!20}{$\widehat{y}\,,r$}}
\newcommand{\teclafatorial}{\tecla{black}{blue!20}{n!}}

% Quarta linha da HP-12C (excluindo enter, inclindo lstx)
\newcommand{\teclaon}{\tecla{black}{white}{ON}}
\newcommand{\teclaf}{\tecla{black}{orange!40}{f}}
\newcommand{\teclag}{\tecla{black}{blue!20}{g}}
\newcommand{\teclasto}{\tecla{black}{white}{STO}}
\newcommand{\teclarcl}{\tecla{black}{white}{RCL}}
\newcommand{\teclazero}{\tecla{black}{white}{0}}
\newcommand{\teclaponto}{\tecla{black}{white}{\tiny{\textbullet}}}
\newcommand{\teclasomamais}{\tecla{black}{white}{$\sum+$}}
\newcommand{\teclamais}{\tecla{black}{white}{$+$}}
\newcommand{\teclalstx}{\tecla{black}{blue!20}{LST\,$x$}}
\newcommand{\teclamediax}{\tecla{black}{blue!20}{$\overline{x}$}}
\newcommand{\teclas}{\tecla{black}{blue!20}{s}}
\newcommand{\teclasomamenos}{\tecla{black}{blue!20}{$\sum-$}}

% Combinações de uso freqüente
\newcommand{\limpaest}{\teclaclear\teclasoma}
\newcommand{\limpaprgm}{\teclaclear\teclaprgm}
\newcommand{\limpafin}{\teclaclear\teclafin}
\newcommand{\limpareg}{\teclaclear\teclareg}
\newcommand{\limpaprefix}{\teclaclear\teclaprefix}
\newcommand{\flimpaest}{\teclaf\teclaclear\teclasoma}
\newcommand{\flimpaprgm}{\teclaf\teclaclear\teclaprgm}
\newcommand{\flimpafin}{\teclaf\teclaclear\teclafin}
\newcommand{\flimpareg}{\teclaf\teclaclear\teclareg}
\newcommand{\flimpaprefix}{\teclaf\teclaclear\teclaprefix}

% Para o conteúdo da pilha operacional
\newcommand{\cs}[1]{\emph{#1}}
\newcommand{\cv}[1]{\textcolor{red}{#1}}

% Para a Platinum/Prestige
\newcommand{\teclaigual}{\tecla{black}{blue!20}{$=$}}
\newcommand{\teclaxquadrado}{\tecla{black}{blue!20}{$x^2$}}
\newcommand{\teclasetacurva}{\tecla{black}{blue!20}{$\curvearrowleft$}}
\newcommand{\teclasetaesquerda}{\tecla{black}{blue!20}{$\leftarrow$}}
\newcommand{\teclaabrepar}{\tecla{black}{blue!20}{$($}}
\newcommand{\teclafechapar}{\tecla{black}{blue!20}{$)$}}
\newcommand{\teclarpn}{\tecla{black}{orange!40}{RPN}}
\newcommand{\teclaalg}{\tecla{black}{orange!40}{ALG}}
