%%%%%%%%%%%%%%%%%%%%%%%%%%%%%%%%%%%%%%%%%%%%%%%%%%%%%%%%%%%%%%%%%%%%%%%%%%%%%%%%
% Por: Abrantes Araújo Silva Filho
%      abrantesasf@gmail.com
% URL: https://github.com/abrantesasf/latex


%%%%%%%%%%%%%%%%%%%%%%%%%%%%%%%%%%%%%%%%%%%%%%%%%%%%%%%%%%%%%%%%%%%%%%%%%%%%%%%%
%%% Carrega bibliotecas e símbolos matemáticos, fontes adicionais e configura
%%% algumas outras opções
\usepackage{amsmath}
\usepackage{amssymb}
\usepackage{amsthm}
\usepackage{amsfonts}
\usepackage{mathrsfs}
\usepackage{proof}
\usepackage{siunitx}
  \sisetup{group-separator = {\,}}
  \sisetup{group-digits = {integer}}
  \sisetup{output-decimal-marker = {,}}
  \sisetup{separate-uncertainty}
  \sisetup{multi-part-units = single}
  \sisetup{binary-units = true}
\usepackage{bm}
\usepackage{cancel}

% Altera separador decimal via comando, se necessário (prefira o siunitx):
%\mathchardef\period=\mathcode`.
%\DeclareMathSymbol{.}{\mathord}{letters}{"3B}

\usepackage{esvect}
\usepackage{mathtools}

%%%%%%%%%%%%%%%%%%%%%%%%%%%%%%%%%%%%%%%%%%%%%%%%%%%%%%%%%%%%%%%%%%%%%%%%%%%%%%%%
%%% Definições para teoremas, etc.
\makeatletter
\@ifclassloaded{article}{
   \theoremstyle{definition}
   \newtheorem{definicao}{Definição}[section]
   \newtheorem{conjecture}{Conjectura}[section]
   \newtheorem{teorema}{Teorema}[section]
   \newtheorem{lemma}{Lema}[section]
   \newtheorem{corolario}{Corolário}[section]
   \theoremstyle{remark}
   \newtheorem*{nota}{Nota}
   \newtheorem*{observacao}{Observação:}
}{}
\@ifclassloaded{book}{
   \theoremstyle{definition}
   \newtheorem{definicao}{Definição}[chapter]
   \newtheorem{conjecture}{Conjectura}[chapter]
   \newtheorem{teorema}{Teorema}[chapter]
   \newtheorem{lemma}{Lema}[chapter]
   \newtheorem{corolario}{Corolário}[chapter]
   \theoremstyle{remark}
   \newtheorem*{nota}{Nota}
   \newtheorem*{observacao}{Observação:}
}{}
\makeatother
