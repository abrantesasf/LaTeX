%%%%%%%%%%%%%%%%%%%%%%%%%%%%%%%%%%%%%%%%%%%%%%%%%%%%%%%%%%%%%%%%%%%%%%%%%%%%%%%%
% Por: Abrantes Araújo Silva Filho
%      abrantesasf@gmail.com
% URL: https://github.com/abrantesasf/latex


%%%%%%%%%%%%%%%%%%%%%%%%%%%%%%%%%%%%%%%%%%%%%%%%%%%%%%%%%%%%%%%%%%%%%%%%%%%%%%%%
%%% Estruturas de controle:
%%%%%%%%%%%%%%%%%%%%%%%%%%%%%%%%%%%%%%%%%%%%%%%%%%%%%%%%%%%%%%%%%%%%%%%%%%%%%%%%
% Por: Abrantes Araújo Silva Filho
%      abrantesasf@gmail.com
% URL: https://github.com/abrantesasf/latex


%%%%%%%%%%%%%%%%%%%%%%%%%%%%%%%%%%%%%%%%%%%%%%%%%%%%%%%%%%%%%%%%%%%%%%%%%%%%%%%%
%%% Carrega pacotes iniciais necessários para estrutura de controle e para a
%%% criação e o parse de novos comandos
\usepackage{ifthen}
\usepackage{xparse}
\usepackage{ifxetex}


%%%%%%%%%%%%%%%%%%%%%%%%%%%%%%%%%%%%%%%%%%%%%%%%%%%%%%%%%%%%%%%%%%%%%%%%%%%%%%%%
%%% Compilação condicional em PDF:
%%%%%%%%%%%%%%%%%%%%%%%%%%%%%%%%%%%%%%%%%%%%%%%%%%%%%%%%%%%%%%%%%%%%%%%%%%%%%%%%
% Por: Abrantes Araújo Silva Filho
%      abrantesasf@gmail.com
% URL: https://github.com/abrantesasf/latex


%%%%%%%%%%%%%%%%%%%%%%%%%%%%%%%%%%%%%%%%%%%%%%%%%%%%%%%%%%%%%%%%%%%%%%%%%%%%%%%%
%%% Pacotes para compilaçãocondicional em PDF:
\ifxetex
\else
   \usepackage{ifpdf}
\fi


%%%%%%%%%%%%%%%%%%%%%%%%%%%%%%%%%%%%%%%%%%%%%%%%%%%%%%%%%%%%%%%%%%%%%%%%%%%%%%%%
%%% Configurações de layout da página:
%%%%%%%%%%%%%%%%%%%%%%%%%%%%%%%%%%%%%%%%%%%%%%%%%%%%%%%%%%%%%%%%%%%%%%%%%%%%%%%%
% Por: Abrantes Araújo Silva Filho
%      abrantesasf@gmail.com
% URL: https://github.com/abrantesasf/latex


%%%%%%%%%%%%%%%%%%%%%%%%%%%%%%%%%%%%%%%%%%%%%%%%%%%%%%%%%%%%%%%%%%%%%%%%%%%%%%%%
%%% Configuração do tamanho da página, margens, espaçamento entrelinhas e, se
%%% necessário, ativa a indentação dos primeiros parágrafos.

\makeatletter
\@ifclassloaded{beamer}{}{
\usepackage{setspace}
\ifxetex
   \usepackage{geometry}
\else
   \ifpdf
     \usepackage[pdftex]{geometry}
   \else
     \usepackage[dvips]{geometry}
   \fi
\fi
}
\makeatother


%%%%%%%%%%%%%%%%%%%%%%%%%%%%%%%%%%%%%%%%%%%%%%%%%%%%%%%%%%%%%%%%%%%%%%%%%%%%%%%%
%%% Configurações para autores:
%%%%%%%%%%%%%%%%%%%%%%%%%%%%%%%%%%%%%%%%%%%%%%%%%%%%%%%%%%%%%%%%%%%%%%%%%%%%%%%%
% Por: Abrantes Araújo Silva Filho
%      abrantesasf@gmail.com
% URL: https://github.com/abrantesasf/latex


%%%%%%%%%%%%%%%%%%%%%%%%%%%%%%%%%%%%%%%%%%%%%%%%%%%%%%%%%%%%%%%%%%%%%%%%%%%%%%%%
%%% Ajustes para os autores e afiliações, em artigos
\makeatletter
\@ifclassloaded{article}{
\usepackage{authblk}
}{}
\makeatother


%%%%%%%%%%%%%%%%%%%%%%%%%%%%%%%%%%%%%%%%%%%%%%%%%%%%%%%%%%%%%%%%%%%%%%%%%%%%%%%%
%%% Cabeçalho e rodapé:
%%% Atenção! É MUITO DIFÍCIL ajustar apropriadamente os running headers
%%% e running footers da primeira vez. Você pode confiar no LaTeX e deixar que
%%% ele faça o ajuste automaticamente ou pode quebrar a cabeça e
%%% tentar melhorar algo que o LaTeX já faz muito bem, mesmo que não fique
%%% exatamente do jeito que você quer. O que você prefere? Se quiser ajustar
%%% manualmente, descomente a linha abaixo e faça as configurações no arquivo
%%% "utils/headings.tex".
%%%%%%%%%%%%%%%%%%%%%%%%%%%%%%%%%%%%%%%%%%%%%%%%%%%%%%%%%%%%%%%%%%%%%%%%%%%%%%%%%
% headings.tex
%
% Arquivo de configuração de packages para uso o LaTeX, conforme minhas
% preferências e modelos pessoais.
%
% Para maiores informações, visite:
%    https://github.com/abrantesasf/latex
%
% NÃO ALTERE SE NÃO SOUBER O QUE ESTÁ FAZENDO!
%%%%%%%%%%%%%%%%%%%%%%%%%%%%%%%%%%%%%%%%%%%%%%%%%%%%%%%%%%%%%%%%%%%%%%%%%%%%%%%%


%%%%%%%%%%%%%%%%%%%%%%%%%%%%%%%%%%%%%%%%%%%%%%%%%%%%%%%%%%%%%%%%%%%%%%%%%%%%%%%%
%%% Configurações de cabeçalho e rodapé:
\makeatletter
\@ifclassloaded{exam}{}{
  \usepackage{fancyhdr}
  \setlength{\headheight}{1cm}
  \setlength{\footskip}{1.5cm}
  \renewcommand{\headrulewidth}{0.3pt}
  \renewcommand{\footrulewidth}{0.0pt}
  \pagestyle{fancy}
  \renewcommand{\sectionmark}[1]{%
    \markboth{\uppercase{#1}}{}
  }
  \renewcommand{\subsectionmark}[1]{%
    \markright{\uppercase{\thesubsection \hspace{0.1cm} #1}}{}
  }
  \fancyhead{}
  \fancyfoot{}
  \newcommand{\diminuifonte}{%
    \fontsize{9pt}{9}\selectfont
  }
  \newcommand{\aumentafonte}{%
    \fontsize{12}{12}\selectfont
  }
  % Configura cabeçalho e rodapé para documentos TWOSIDE
  \fancyhead[EL]{\textbf{\thepage}}
  \fancyhead[EC]{}
  \fancyhead[ER]{\diminuifonte \textbf{\leftmark}}
  \fancyhead[OR]{\textbf{\thepage}}
  \fancyhead[OC]{}
  \fancyhead[OL]{\diminuifonte \textbf{\rightmark}}
  \fancyfoot[EL,EC,ER,OR,OC,OL]{}
  % Configura cabeçalho e rodapé para documentos ONESIDE
  %%\lhead{ \fancyplain{}{sup esquerdo} }
  %%\chead{ \fancyplain{}{sup centro} }
  %%\rhead{ \fancyplain{}{\thesection} }
  %%\lfoot{ \fancyplain{}{inf esquerdo} }
  %%\cfoot{ \fancyplain{}{inf centro} }
  %%\rfoot{ \fancyplain{}{\thepage} }
}
\makeatother



%%%%%%%%%%%%%%%%%%%%%%%%%%%%%%%%%%%%%%%%%%%%%%%%%%%%%%%%%%%%%%%%%%%%%%%%%%%%%%%%
%%% Configuração para as fontes do documento:
%%% Se você quiser usar fontes próprias ou do sistema, precisará ajustar as
%%% configurações no arquivo "utils/fontes.tex".
%%% ATENÇÃO: por padrão eu utilizo XeLaTeX com fontes proprietárias projetadas
%%% por Matthew Butterick, adquiridas em https://mbtype.com, que não podem ser
%%% distribuídas devido à licença de utilização. Se você usar XeLaTeX, você
%%% DEVERÁ OBRIGATORIAMENTE ajustar as fontes no arquivo "utils/fontes.tex".
%%%%%%%%%%%%%%%%%%%%%%%%%%%%%%%%%%%%%%%%%%%%%%%%%%%%%%%%%%%%%%%%%%%%%%%%%%%%%%%%
% Por: Abrantes Araújo Silva Filho
%      abrantesasf@gmail.com
% URL: https://github.com/abrantesasf/latex


%%%%%%%%%%%%%%%%%%%%%%%%%%%%%%%%%%%%%%%%%%%%%%%%%%%%%%%%%%%%%%%%%%%%%%%%%%%%%%%%
%%% Configurações de encodings e fontes:
\ifxetex
   % Se usar XeLaTeX, usa fontes específicas:
   \usepackage[tuenc,no-math]{fontspec}
   \setmainfont{equity-text-a-regular.otf}[
      Path           = /home/abrantesasf/.local/share/fonts/ ,
      BoldFont       = equity-text-a-bold.otf                ,
      ItalicFont     = equity-text-a-italic.otf              ,
      BoldItalicFont = equity-text-a-bold-italic.otf]
\else
   % Se não for XeLaTeX, vai com as normais mesmo:
   \usepackage[T1]{fontenc}
   \usepackage[utf8]{inputenc}
   % Altera a fonte padrão do documento (nem todas funcionam em modo math):
   %   phv = Helvetica
   %   ptm = Times
   %   ppl = Palatino
   %   pbk = bookman
   %   pag = AdobeAvantGarde
   %   pnc = Adobe NewCenturySchoolBook
   \renewcommand{\familydefault}{ppl}
\fi


%%%%%%%%%%%%%%%%%%%%%%%%%%%%%%%%%%%%%%%%%%%%%%%%%%%%%%%%%%%%%%%%%%%%%%%%%%%%%%%%
%%% Linguagem e hifenização:
%%%%%%%%%%%%%%%%%%%%%%%%%%%%%%%%%%%%%%%%%%%%%%%%%%%%%%%%%%%%%%%%%%%%%%%%%%%%%%%%
% Por: Abrantes Araújo Silva Filho
%      abrantesasf@gmail.com
% URL: https://github.com/abrantesasf/latex


%%%%%%%%%%%%%%%%%%%%%%%%%%%%%%%%%%%%%%%%%%%%%%%%%%%%%%%%%%%%%%%%%%%%%%%%%%%%%%%%
%%% Configurações de linguagem e hifenização
\usepackage{polyglossia}
\setdefaultlanguage[variant=brazilian]{portuguese}
\setotherlanguage{english}

%%%%%%%%%%%%%%%%%%%%%%%%%%%%%%%%%%%%%%%%%%%%%%%%%%%%%%%%%%%%%%%%%%%%%%%%%%%%%%%%
%%% Hifenização específica de alguams palavras quando a hifenização padrão não
%%% estiver hifenizando corretamente:
\hyphenation{%
Git-Hub
LaTeX
po-ly-glos-sia
}


%%%%%%%%%%%%%%%%%%%%%%%%%%%%%%%%%%%%%%%%%%%%%%%%%%%%%%%%%%%%%%%%%%%%%%%%%%%%%%%%
%%% Sumário:
%%%%%%%%%%%%%%%%%%%%%%%%%%%%%%%%%%%%%%%%%%%%%%%%%%%%%%%%%%%%%%%%%%%%%%%%%%%%%%%%
% Por: Abrantes Araújo Silva Filho
%      abrantesasf@gmail.com
% URL: https://github.com/abrantesasf/latex


%%%%%%%%%%%%%%%%%%%%%%%%%%%%%%%%%%%%%%%%%%%%%%%%%%%%%%%%%%%%%%%%%%%%%%%%%%%%%%%%
%%% Ajustes de sumário:
\makeatletter
\@ifclassloaded{article}{
   \usepackage[nottoc]{tocbibind}
   %\usepackage{tocbibind}
}{}
\@ifclassloaded{book}{
   \usepackage{tocbibind}
}{}
\makeatother


%%%%%%%%%%%%%%%%%%%%%%%%%%%%%%%%%%%%%%%%%%%%%%%%%%%%%%%%%%%%%%%%%%%%%%%%%%%%%%%%
%%% Matemática:
%%%%%%%%%%%%%%%%%%%%%%%%%%%%%%%%%%%%%%%%%%%%%%%%%%%%%%%%%%%%%%%%%%%%%%%%%%%%%%%%
% matematica.tex
%
% Arquivo de configuração de packages para uso o LaTeX, conforme minhas
% preferências e modelos pessoais.
%
% Para maiores informações, visite:
%    https://github.com/abrantesasf/latex
%
% NÃO ALTERE SE NÃO SOUBER O QUE ESTÁ FAZENDO!
%%%%%%%%%%%%%%%%%%%%%%%%%%%%%%%%%%%%%%%%%%%%%%%%%%%%%%%%%%%%%%%%%%%%%%%%%%%%%%%%


%%%%%%%%%%%%%%%%%%%%%%%%%%%%%%%%%%%%%%%%%%%%%%%%%%%%%%%%%%%%%%%%%%%%%%%%%%%%%%%%
%%% Carrega bibliotecas e símbolos matemáticos, fontes adicionais e configura
%%% algumas outras opções
\usepackage{amsmath}
\usepackage{amssymb}
\usepackage{amsthm}
\usepackage{amsfonts}
\usepackage{mathrsfs}
\usepackage{proof}
\usepackage{siunitx}
  \sisetup{group-separator = {\,}}
  \sisetup{group-digits = {integer}}
  \sisetup{output-decimal-marker = {,}}
  \sisetup{separate-uncertainty}
  \sisetup{multi-part-units = single}
  \sisetup{binary-units = true}
  \sisetup{list-final-separator = { e }}
\usepackage{syllogism}
  \setsylpuncpa{}
  \setsylpuncpb{}
  \setsylpuncc{}
  \setsylergosign{}
\usepackage{bm}
\usepackage{cancel}
\usepackage{esvect}
\usepackage{mathtools}
\usepackage{icomma}
\usepackage{nicefrac}
%\usepackage{units}

% Altera separador decimal via comando, se necessário (prefira o siunitx):
%\mathchardef\period=\mathcode`.
%\DeclareMathSymbol{.}{\mathord}{letters}{"3B}

% Declara alguams unidades adicionais para siunitx:
\DeclareSIUnit{\degreeFahrenheit}{\unit{\degree}F}
\DeclareSIUnit{\gravity}{\mbox{$g$}}


%%%%%%%%%%%%%%%%%%%%%%%%%%%%%%%%%%%%%%%%%%%%%%%%%%%%%%%%%%%%%%%%%%%%%%%%%%%%%%%%
%%% Definições para teoremas, etc.

% Para article:
\makeatletter
\@ifclassloaded{article}{
  \theoremstyle{definition}
  \newtheorem{definicao}{Definição}[section]
  \newtheorem{conjecture}{Conjectura}[section]
  \newtheorem{teorema}{Teorema}[section]
  \newtheorem{lemma}{Lema}[section]
  \newtheorem{corolario}{Corolário}[section]
  \theoremstyle{remark}
  \newtheorem*{nota}{Nota}
  \newtheorem*{observacao}{Observação:}
}{}

% Para book:
\@ifclassloaded{book}{
  \theoremstyle{definition}
  \newtheorem{definicao}{Definição}[chapter]
  \newtheorem{conjecture}{Conjectura}[chapter]
  \newtheorem{teorema}{Teorema}[chapter]
  \newtheorem{lemma}{Lema}[chapter]
  \newtheorem{corolario}{Corolário}[chapter]
  \theoremstyle{remark}
  \newtheorem*{nota}{Nota}
  \newtheorem*{observacao}{Observação:}
}{}
\makeatother



%%%%%%%%%%%%%%%%%%%%%%%%%%%%%%%%%%%%%%%%%%%%%%%%%%%%%%%%%%%%%%%%%%%%%%%%%%%%%%%%
%%% Notas de rodapé:
%%%%%%%%%%%%%%%%%%%%%%%%%%%%%%%%%%%%%%%%%%%%%%%%%%%%%%%%%%%%%%%%%%%%%%%%%%%%%%%%
% Por: Abrantes Araújo Silva Filho
%      abrantesasf@gmail.com
% URL: https://github.com/abrantesasf/latex


%%%%%%%%%%%%%%%%%%%%%%%%%%%%%%%%%%%%%%%%%%%%%%%%%%%%%%%%%%%%%%%%%%%%%%%%%%%%%%%%%
%%% Carrega pacotes para maior controle de notas de rodapé, inclusive dando
%%% um label para uma nota e fazendo referência cruzada.
%%% NÃO ALTERE a ordem de carregamento dos packages!
\usepackage{footmisc}


%%%%%%%%%%%%%%%%%%%%%%%%%%%%%%%%%%%%%%%%%%%%%%%%%%%%%%%%%%%%%%%%%%%%%%%%%%%%%%%%
%%% Referências cruzadas, links, citações:
%%%%%%%%%%%%%%%%%%%%%%%%%%%%%%%%%%%%%%%%%%%%%%%%%%%%%%%%%%%%%%%%%%%%%%%%%%%%%%%%
% Por: Abrantes Araújo Silva Filho
%      abrantesasf@gmail.com
% URL: https://github.com/abrantesasf/latex


%%%%%%%%%%%%%%%%%%%%%%%%%%%%%%%%%%%%%%%%%%%%%%%%%%%%%%%%%%%%%%%%%%%%%%%%%%%%%%%%%
%%% Carrega pacotes para referências cruzadas, citações dentro do documento,
%%% links para internet e outros.Configura algumas opções.
%%% Não altere a ordem de carregamento dos packages.
\makeatletter
  \@ifclassloaded{beamer}{
}{
  \usepackage{varioref}
  \ifpdf
    \usepackage[pdftex]{hyperref}
      \hypersetup{
        % Configurações padrão que eu gosto
        unicode=true,
        pdflang={pt-BR},
        bookmarksopen=true,
        bookmarksnumbered=true,
        bookmarksopenlevel=5,
        pdfdisplaydoctitle=true,
        pdfpagemode=UseOutlines,
        pdfstartview=FitH,
        pdfcreator={LaTeX with hyperref package},
        pdfproducer={pdfTeX},
        pdfnewwindow=true,
        colorlinks=true,
        citecolor=red,
        linkcolor=red,
        filecolor=cyan,
        urlcolor=blue
      }
  \else
    \usepackage{hyperref}
  \fi
  \usepackage{cleveref}
  \usepackage{url}
}
\makeatother


%%%%%%%%%%%%%%%%%%%%%%%%%%%%%%%%%%%%%%%%%%%%%%%%%%%%%%%%%%%%%%%%%%%%%%%%%%%%%%%%
%%% Glossário e índice remissivo:
%%%%%%%%%%%%%%%%%%%%%%%%%%%%%%%%%%%%%%%%%%%%%%%%%%%%%%%%%%%%%%%%%%%%%%%%%%%%%%%%
% glossindex.tex
%
% Arquivo de configuração de packages para uso o LaTeX, conforme minhas
% preferências e modelos pessoais.
%
% Para maiores informações, visite:
%    https://github.com/abrantesasf/latex
%
% NÃO ALTERE SE NÃO SOUBER O QUE ESTÁ FAZENDO!
%%%%%%%%%%%%%%%%%%%%%%%%%%%%%%%%%%%%%%%%%%%%%%%%%%%%%%%%%%%%%%%%%%%%%%%%%%%%%%%%


%%%%%%%%%%%%%%%%%%%%%%%%%%%%%%%%%%%%%%%%%%%%%%%%%%%%%%%%%%%%%%%%%%%%%%%%%%%%%%%%
%%% Pacote para índice remissivo
\makeatletter
\@ifclassloaded{book}{
  \usepackage{makeidx}
  \makeindex
}{}
\makeatother


%%%%%%%%%%%%%%%%%%%%%%%%%%%%%%%%%%%%%%%%%%%%%%%%%%%%%%%%%%%%%%%%%%%%%%%%%%%%%%%%
%%% Pacote para glossário
\makeatletter
\@ifclassloaded{book}{
  \usepackage[toc]{glossaries}
  %\newglossary[nlg]{notation}{not}{ntn}{Notação}
  %\newglossaryentry{not:set}{
  %   type = notation,
  %   name = {$\mathbb{N}$},
  %   description = conjunto dos números naturais,
  %   sort = {N}}
  %\makeglossaries
}{}
\makeatother



%%%%%%%%%%%%%%%%%%%%%%%%%%%%%%%%%%%%%%%%%%%%%%%%%%%%%%%%%%%%%%%%%%%%%%%%%%%%%%%%
%%% Referências bibliográficas:
%%%%%%%%%%%%%%%%%%%%%%%%%%%%%%%%%%%%%%%%%%%%%%%%%%%%%%%%%%%%%%%%%%%%%%%%%%%%%%%%
% bibliografia.tex
%
% Arquivo de configuração de packages para uso o LaTeX, conforme minhas
% preferências e modelos pessoais.
%
% Para maiores informações, visite:
%    https://github.com/abrantesasf/latex
%
% NÃO ALTERE SE NÃO SOUBER O QUE ESTÁ FAZENDO!
%%%%%%%%%%%%%%%%%%%%%%%%%%%%%%%%%%%%%%%%%%%%%%%%%%%%%%%%%%%%%%%%%%%%%%%%%%%%%%%%


%%%%%%%%%%%%%%%%%%%%%%%%%%%%%%%%%%%%%%%%%%%%%%%%%%%%%%%%%%%%%%%%%%%%%%%%%%%%%%%%
%%% Referências bibliográficas
\usepackage[brazilian,hyperpageref]{backref} % Páginas onde foi citado
\usepackage[alf]{abntex2cite}	             % Citações padrão ABNT

%%% Configurações do pacote backref:
\renewcommand{\backrefpagesname}{Citado na(s) página(s):~}
\renewcommand{\backref}{}
\renewcommand*{\backrefalt}[4]{
  \ifcase #1 %
    Nenhuma citação no texto.%
  \or
    Citado na página #2.%
  \else
    Citado #1 vezes, nas páginas #2.%
  \fi%
}%


%%%%%%%%%%%%%%%%%%%%%%%%%%%%%%%%%%%%%%%%%%%%%%%%%%%%%%%%%%%%%%%%%%%%%%%%%%%%%%%%
%%% Cores:
%%%%%%%%%%%%%%%%%%%%%%%%%%%%%%%%%%%%%%%%%%%%%%%%%%%%%%%%%%%%%%%%%%%%%%%%%%%%%%%%
% Por: Abrantes Araújo Silva Filho
%      abrantesasf@gmail.com
% URL: https://github.com/abrantesasf/latex


%%%%%%%%%%%%%%%%%%%%%%%%%%%%%%%%%%%%%%%%%%%%%%%%%%%%%%%%%%%%%%%%%%%%%%%%%%%%%%%%
%%% Ativa suporte extendido a cores
\makeatletter
\@ifclassloaded{beamer}{
  \usepackage{xcolor} % Opções de cores: usenames (16), dvipsnames (64),
                      % svgnames (150) e x11names (300).
}{
  \usepackage[svgnames]{xcolor} % Opções de cores: usenames (16), dvipsnames (64),
                                % svgnames (150) e x11names (300).
}
\makeatother


%%%%%%%%%%%%%%%%%%%%%%%%%%%%%%%%%%%%%%%%%%%%%%%%%%%%%%%%%%%%%%%%%%%%%%%%%%%%%%%%
%%% Computação:
%%%%%%%%%%%%%%%%%%%%%%%%%%%%%%%%%%%%%%%%%%%%%%%%%%%%%%%%%%%%%%%%%%%%%%%%%%%%%%%%
% computacao.tex
%
% Arquivo de configuração de packages para uso o LaTeX, conforme minhas
% preferências e modelos pessoais.
%
% Para maiores informações, visite:
%    https://github.com/abrantesasf/latex
%
% NÃO ALTERE SE NÃO SOUBER O QUE ESTÁ FAZENDO!
%%%%%%%%%%%%%%%%%%%%%%%%%%%%%%%%%%%%%%%%%%%%%%%%%%%%%%%%%%%%%%%%%%%%%%%%%%%%%%%%


%%%%%%%%%%%%%%%%%%%%%%%%%%%%%%%%%%%%%%%%%%%%%%%%%%%%%%%%%%%%%%%%%%%%%%%%%%%%%%%%
%%% Carrega packages relacionados à computação
\usepackage{algorithm2e}
\usepackage{algorithmicx}
\usepackage{algpseudocode}
\usepackage{listings}
  \lstset{literate=
    {á}{{\'a}}1 {é}{{\'e}}1 {í}{{\'i}}1 {ó}{{\'o}}1 {ú}{{\'u}}1
    {Á}{{\'A}}1 {É}{{\'E}}1 {Í}{{\'I}}1 {Ó}{{\'O}}1 {Ú}{{\'U}}1
    {à}{{\`a}}1 {è}{{\`e}}1 {ì}{{\`i}}1 {ò}{{\`o}}1 {ù}{{\`u}}1
    {À}{{\`A}}1 {È}{{\'E}}1 {Ì}{{\`I}}1 {Ò}{{\`O}}1 {Ù}{{\`U}}1
    {ä}{{\"a}}1 {ë}{{\"e}}1 {ï}{{\"i}}1 {ö}{{\"o}}1 {ü}{{\"u}}1
    {Ä}{{\"A}}1 {Ë}{{\"E}}1 {Ï}{{\"I}}1 {Ö}{{\"O}}1 {Ü}{{\"U}}1
    {â}{{\^a}}1 {ê}{{\^e}}1 {î}{{\^i}}1 {ô}{{\^o}}1 {û}{{\^u}}1
    {Â}{{\^A}}1 {Ê}{{\^E}}1 {Î}{{\^I}}1 {Ô}{{\^O}}1 {Û}{{\^U}}1
    {œ}{{\oe}}1 {Œ}{{\OE}}1 {æ}{{\ae}}1 {Æ}{{\AE}}1 {ß}{{\ss}}1
    {ű}{{\H{u}}}1 {Ű}{{\H{U}}}1 {ő}{{\H{o}}}1 {Ő}{{\H{O}}}1
    {ç}{{\c c}}1 {Ç}{{\c C}}1 {ø}{{\o}}1 {å}{{\r a}}1 {Å}{{\r A}}1
    {€}{{\euro}}1 {£}{{\pounds}}1 {«}{{\guillemotleft}}1
    {»}{{\guillemotright}}1 {ñ}{{\~n}}1 {Ñ}{{\~N}}1 {¿}{{?`}}1
  }
\definecolor{mGreen}{rgb}{0,0.6,0}
\definecolor{mGray}{rgb}{0.5,0.5,0.5}
\definecolor{mPurple}{rgb}{0.58,0,0.82}
\definecolor{mymauve}{rgb}{0.58,0,0.82}
\definecolor{backgroundColour}{rgb}{0.95,0.95,0.92}

\lstdefinestyle{CStyle}{
    backgroundcolor=\color{backgroundColour},   
    commentstyle=\color{green}\ttfamily,
    keywordstyle=\color{blue}\ttfamily,
    numberstyle=\tiny\color{mGray},
    stringstyle=\color{red}\ttfamily,
    basicstyle=\ttfamily,
    morecomment=[l][\color{magenta}]{\#},
    breakatwhitespace=false,         
    breaklines=true,                 
    captionpos=b,                    
    keepspaces=true,                 
    numbers=left,                    
    numbersep=5pt,                  
    showspaces=false,                
    showstringspaces=false,
    showtabs=false,                  
    tabsize=4,
    language=C
}

%\lstset{ 
%    backgroundcolor=\color{white},     % choose the background color; you must add \usepackage{color} or
%                                       %\usepackage{xcolor}; should come as last argument
%    basicstyle=\footnotesize\ttfamily, % the size of the fonts that are used for the code
%    breakatwhitespace=false,           % sets if automatic breaks should only happen at whitespace
%    breaklines=true,                   % sets automatic line breaking
%    captionpos=b,                      % sets the caption-position to bottom
%    commentstyle=\color{mygreen},      % comment style
%    deletekeywords={...},              % if you want to delete keywords from the given language
%    escapeinside={\%*}{*)},            % if you want to add LaTeX within your code
%    extendedchars=true,                % lets you use non-ASCII characters; for 8-bits encodings only,
%                                       % does not work with UTF-8
%    firstnumber=0,                     % start line enumeration with line 1000
%    frame=single,	                   % adds a frame around the code
%    keepspaces=true,                   % keeps spaces in text, useful for keeping indentation of code
%                                       % (possibly needs columns=flexible)
%    keywordstyle=\color{blue},         % keyword style
%    language=Octave,                   % the language of the code
%    morekeywords={*,...},              % if you want to add more keywords to the set
%    numbers=left,                      % where to put the line-numbers; possible values are (none, left, right)
%    numbersep=5pt,                     % how far the line-numbers are from the code
%    numberstyle=\tiny\color{mygray},   % the style that is used for the line-numbers
%    rulecolor=\color{black},           % if not set, the frame-color may be changed on line-breaks
%                                       % within not-black text (e.g. comments (green here))
%    showspaces=false,                  % show spaces everywhere adding particular underscores;
%                                       % it overrides 'showstringspaces'
%    showstringspaces=false,            % underline spaces within strings only
%    showtabs=false,                    % show tabs within strings adding particular underscores
%    stepnumber=2,                      % the step between two line-numbers. If it's 1, each line will be numbered
%    stringstyle=\color{mymauve},       % string literal style
%    tabsize=2,	                       % sets default tabsize to 2 spaces
%    %title=\lstname,                    % show the filename of files included with \lstinputlisting;
%                                       % also try caption instead of title
%    xleftmargin=0.25cm
%}


%%%%%%%%%%%%%%%%%%%%%%%%%%%%%%%%%%%%%%%%%%%%%%%%%%%%%%%%%%%%%%%%%%%%%%%%%%%%%%%%
%%% Gráficos:
%%%%%%%%%%%%%%%%%%%%%%%%%%%%%%%%%%%%%%%%%%%%%%%%%%%%%%%%%%%%%%%%%%%%%%%%%%%%%%%%
% Por: Abrantes Araújo Silva Filho
%      abrantesasf@gmail.com
% URL: https://github.com/abrantesasf/latex


%%%%%%%%%%%%%%%%%%%%%%%%%%%%%%%%%%%%%%%%%%%%%%%%%%%%%%%%%%%%%%%%%%%%%%%%%%%%%%%%
%%% Suporte à importação de gráficos externos
\makeatletter
\@ifclassloaded{beamer}{
   \usepackage{graphicx}
}{
   \ifxetex
      \usepackage{graphicx}
   \else
      \ifpdf
         \usepackage[pdftex]{graphicx}
      \else
         \usepackage[dvips]{graphicx}
      \fi
   \fi
}
\makeatother


%%%%%%%%%%%%%%%%%%%%%%%%%%%%%%%%%%%%%%%%%%%%%%%%%%%%%%%%%%%%%%%%%%%%%%%%%%%%%%%%
%%% Suporte à criação de gráficos proceduralmente no LaTeX:
\usepackage{tikz}
\usetikzlibrary{arrows,automata,backgrounds,matrix,patterns,positioning,shapes,shadows}


%%%%%%%%%%%%%%%%%%%%%%%%%%%%%%%%%%%%%%%%%%%%%%%%%%%%%%%%%%%%%%%%%%%%%%%%%%%%%%%%
%%% Ícones:
%%%%%%%%%%%%%%%%%%%%%%%%%%%%%%%%%%%%%%%%%%%%%%%%%%%%%%%%%%%%%%%%%%%%%%%%%%%%%%%%
% icones.tex
%
% Arquivo de configuração de packages para uso o LaTeX, conforme minhas
% preferências e modelos pessoais.
%
% Para maiores informações, visite:
%    https://github.com/abrantesasf/latex
%
% NÃO ALTERE SE NÃO SOUBER O QUE ESTÁ FAZENDO!
%%%%%%%%%%%%%%%%%%%%%%%%%%%%%%%%%%%%%%%%%%%%%%%%%%%%%%%%%%%%%%%%%%%%%%%%%%%%%%%%


%%%%%%%%%%%%%%%%%%%%%%%%%%%%%%%%%%%%%%%%%%%%%%%%%%%%%%%%%%%%%%%%%%%%%%%%%%%%%%%%
%%% Pacotes com ícones, figuras e gráficos extras

% Ícones da Creative Commons:
\usepackage{ccicons}



%%%%%%%%%%%%%%%%%%%%%%%%%%%%%%%%%%%%%%%%%%%%%%%%%%%%%%%%%%%%%%%%%%%%%%%%%%%%%%%%
%%% Blocos:
%%%%%%%%%%%%%%%%%%%%%%%%%%%%%%%%%%%%%%%%%%%%%%%%%%%%%%%%%%%%%%%%%%%%%%%%%%%%%%%%
% blocos.tex
%
% Arquivo de configuração de packages para uso o LaTeX, conforme minhas
% preferências e modelos pessoais.
%
% Para maiores informações, visite:
%    https://github.com/abrantesasf/latex
%
% NÃO ALTERE SE NÃO SOUBER O QUE ESTÁ FAZENDO!
%%%%%%%%%%%%%%%%%%%%%%%%%%%%%%%%%%%%%%%%%%%%%%%%%%%%%%%%%%%%%%%%%%%%%%%%%%%%%%%%


%%%%%%%%%%%%%%%%%%%%%%%%%%%%%%%%%%%%%%%%%%%%%%%%%%%%%%%%%%%%%%%%%%%%%%%%%%%%%%%%
%%% Ambiente para boxes em posições arbitrárias no Beamer
\makeatletter
\@ifclassloaded{beamer}{
   \usepackage{textpos}
}{}
\makeatother


%%%%%%%%%%%%%%%%%%%%%%%%%%%%%%%%%%%%%%%%%%%%%%%%%%%%%%%%%%%%%%%%%%%%%%%%%%%%%%%%
%%% Boxes coloridos:
%%%%%%%%%%%%%%%%%%%%%%%%%%%%%%%%%%%%%%%%%%%%%%%%%%%%%%%%%%%%%%%%%%%%%%%%%%%%%%%%
% colorbox.tex
%
% Arquivo de configuração de packages para uso o LaTeX, conforme minhas
% preferências e modelos pessoais.
%
% Para maiores informações, visite:
%    https://github.com/abrantesasf/latex
%
% NÃO ALTERE SE NÃO SOUBER O QUE ESTÁ FAZENDO!
%%%%%%%%%%%%%%%%%%%%%%%%%%%%%%%%%%%%%%%%%%%%%%%%%%%%%%%%%%%%%%%%%%%%%%%%%%%%%%%%


%%%%%%%%%%%%%%%%%%%%%%%%%%%%%%%%%%%%%%%%%%%%%%%%%%%%%%%%%%%%%%%%%%%%%%%%%%%%%%%%
%%% Ativa suporte para boxes coloridos
\usepackage[many]{tcolorbox}


%%%%%%%%%%%%%%%%%%%%%%%%%%%%%%%%%%%%%%%%%%%%%%%%%%%%%%%%%%%%%%%%%%%%%%%%%%%%%%%%
%%% Tabelas:
%%%%%%%%%%%%%%%%%%%%%%%%%%%%%%%%%%%%%%%%%%%%%%%%%%%%%%%%%%%%%%%%%%%%%%%%%%%%%%%%
% tabelas.tex
%
% Arquivo de configuração de packages para uso o LaTeX, conforme minhas
% preferências e modelos pessoais.
%
% Para maiores informações, visite:
%    https://github.com/abrantesasf/latex
%
% NÃO ALTERE SE NÃO SOUBER O QUE ESTÁ FAZENDO!
%%%%%%%%%%%%%%%%%%%%%%%%%%%%%%%%%%%%%%%%%%%%%%%%%%%%%%%%%%%%%%%%%%%%%%%%%%%%%%%%


%%%%%%%%%%%%%%%%%%%%%%%%%%%%%%%%%%%%%%%%%%%%%%%%%%%%%%%%%%%%%%%%%%%%%%%%%%%%%%%%
%%% Packages para tabelas
\usepackage{array}
\usepackage{longtable}
\usepackage{tabularx}
\usepackage{tabu}
\usepackage{lscape}
\usepackage{colortbl}  
\usepackage{booktabs}
\usepackage{multirow}
\usepackage{makecell}



%%%%%%%%%%%%%%%%%%%%%%%%%%%%%%%%%%%%%%%%%%%%%%%%%%%%%%%%%%%%%%%%%%%%%%%%%%%%%%%%
%%% Ambientes de listas:
%%%%%%%%%%%%%%%%%%%%%%%%%%%%%%%%%%%%%%%%%%%%%%%%%%%%%%%%%%%%%%%%%%%%%%%%%%%%%%%%
% Por: Abrantes Araújo Silva Filho
%      abrantesasf@gmail.com
% URL: https://github.com/abrantesasf/latex


%%%%%%%%%%%%%%%%%%%%%%%%%%%%%%%%%%%%%%%%%%%%%%%%%%%%%%%%%%%%%%%%%%%%%%%%%%%%%%%%
%%% Packages para ambientes de listas
\usepackage{enumitem}
\usepackage[ampersand]{easylist}


%%%%%%%%%%%%%%%%%%%%%%%%%%%%%%%%%%%%%%%%%%%%%%%%%%%%%%%%%%%%%%%%%%%%%%%%%%%%%%%%
%%% Ambientes floats e similares:
%%%%%%%%%%%%%%%%%%%%%%%%%%%%%%%%%%%%%%%%%%%%%%%%%%%%%%%%%%%%%%%%%%%%%%%%%%%%%%%%
% floats.tex
%
% Arquivo de configuração de packages para uso o LaTeX, conforme minhas
% preferências e modelos pessoais.
%
% Para maiores informações, visite:
%    https://github.com/abrantesasf/latex
%
% NÃO ALTERE SE NÃO SOUBER O QUE ESTÁ FAZENDO!
%%%%%%%%%%%%%%%%%%%%%%%%%%%%%%%%%%%%%%%%%%%%%%%%%%%%%%%%%%%%%%%%%%%%%%%%%%%%%%%%


%%%%%%%%%%%%%%%%%%%%%%%%%%%%%%%%%%%%%%%%%%%%%%%%%%%%%%%%%%%%%%%%%%%%%%%%%%%%%%%%
%%% Packages para suporte a ambientes floats, captions, etc.:
\usepackage{float}
\usepackage{wrapfig}
\usepackage{placeins}
\usepackage[justification=centering]{caption}
\usepackage{sidecap}
\usepackage{subcaption}



%%%%%%%%%%%%%%%%%%%%%%%%%%%%%%%%%%%%%%%%%%%%%%%%%%%%%%%%%%%%%%%%%%%%%%%%%%%%%%%%
%%% Ambientes, aliases, comandos e customizações em geral para serem
%%% utilizadas nos documentos. Defina aqui qualquer customização específica
%%% para seus documentos!
%%%%%%%%%%%%%%%%%%%%%%%%%%%%%%%%%%%%%%%%%%%%%%%%%%%%%%%%%%%%%%%%%%%%%%%%%%%%%%%%
% customizacoes.tex
%
% Arquivo de configuração de packages para uso o LaTeX, conforme minhas
% preferências e modelos pessoais.
%
% Para maiores informações, visite:
%    https://github.com/abrantesasf/latex
%
% NÃO ALTERE SE NÃO SOUBER O QUE ESTÁ FAZENDO!
%%%%%%%%%%%%%%%%%%%%%%%%%%%%%%%%%%%%%%%%%%%%%%%%%%%%%%%%%%%%%%%%%%%%%%%%%%%%%%%%


%%%%%%%%%%%%%%%%%%%%%%%%%%%%%%%%%%%%%%%%%%%%%%%%%%%%%%%%%%%%%%%%%%%%%%%%%%%%%%%%
%%% Vários comandinhos úteis e outras gambiarras

% Commando para ``italizar´´ palavras em inglês (e outras línguas!):
\newcommand{\ingles}[1]{\textit{#1}}

% Comando para escrever uma função e símbolos em fonte monoespaçada:
\newcommand{\funcao}[1]{\textbf{\texttt{#1}}}
\newcommand{\simbolo}[1]{\texttt{#1}}

% Commando para colocar o espaço correto entre um número e sua unidade:
\newcommand{\unidade}[2]{\ensuremath{#1\,\mathrm{#2}}}
\newcommand{\unidado}[2]{{#1}\,{#2}}

% Produz ordinal masculino ou feminino dependendo do segundo argumento:
\newcommand{\ordinal}[2]{%
#1%
\ifthenelse{\equal{a}{#2}}%
{\textordfeminine}%
{\textordmasculine}}

% Comando para colocar o autor da citação corretamente justificado à direita:
% Modificado de: https://latex.org/forum/viewtopic.php?p=15605&sid=216895679326b531e85caec779e1d710#p15605
%\newcommand*{\fontequote}[2]{%
%   \unskip\hspace*{1em plus 1fill}%
%   \linebreak\hspace*{\fill}\mbox{#1}\vspace{-0.15cm}
%   \linebreak\hspace*{\fill}\mbox{#2}
%}

% Comandos para os quadros padronizados dos livros de matemática:
\newcommand{\abrevideo}[1]{%
\begin{tcolorbox}[colback=yellow!5!white,colframe=yellow!75!black,
       title=\textbf{Vídeo(s) de referência: #1}]}
\newcommand{\fechavideo}{\end{tcolorbox}}

% Comandos para os ícones comuns:
\newcommand{\youtube}{\faYoutube}

% Comandos para álgebra relacional (testados em XeLaTeX):
\def\ojoin{\setbox0=\hbox{$\Join$}%
\rule[0.02ex]{.4em}{.5pt}\llap{\rule[0.95ex]{.4em}{.5pt}}}
\def\leftouterjoin{\mathbin{\ojoin\mkern-7.8mu\Join}}
\def\rightouterjoin{\mathbin{\Join\mkern-7.8mu\ojoin}}
\def\fullouterjoin{\mathbin{\ojoin\mkern-7.8mu\Join\mkern-7.8mu\ojoin}}

% Comando para números dentro de círculos
% https://tex.stackexchange.com/questions/7032/good-way-to-make-textcircled-numbers
\newcommand*\circled[1]{\tikz[baseline=(char.base)]{
            \node[shape=circle,draw,inner sep=2pt] (char) {#1};}}

