%%%%%%%%%%%%%%%%%%%%%%%%%%%%%%%%%%%%%%%%%%%%%%%%%%%%%%%%%%%%%%%%%%%%%%%%%%%%%%%%
% tabelas.tex
%
% Arquivo de configuração de packages para o LaTeX, considerando diversas
% classes que constumo utilizar.
%
% Para maiores informações, visite:
%    https://github.com/abrantesasf/latex
%    https://github.com/uvv-computacao/uvvtex2
%
% NÃO ALTERE SE NÃO SOUBER O QUE ESTÁ FAZENDO!
%%%%%%%%%%%%%%%%%%%%%%%%%%%%%%%%%%%%%%%%%%%%%%%%%%%%%%%%%%%%%%%%%%%%%%%%%%%%%%%%


%%%%%%%%%%%%%%%%%%%%%%%%%%%%%%%%%%%%%%%%%%%%%%%%%%%%%%%%%%%%%%%%%%%%%%%%%%%%%%%%
%%% Packages para tabelas
\usepackage{array}
\usepackage{tabularray}
\DefTblrTemplate{contfoot-text}{default}{\footnotesize (continua na próxima página)}
\DefTblrTemplate{conthead-text}{default}{(continuação)}
\ExplSyntaxOn
\RenewDocumentCommand \TblrNote { m }
  {
    \cs_if_exist:NT \hypersetup
      { \ExpTblrTemplate { note-border }{ default } }
    \TblrOverlap
      {
        \__tblr_hyper_link:nn {#1}
          { \kern0.03em \textsuperscript { \rmfamily \UseTblrFont { note-tag } #1 } }
      }
  }
\ExplSyntaxOff
\DeclareTblrTemplate{note-tag}{normal}{
  \textsuperscript{\rmfamily \UseTblrFont{note-tag}\InsertTblrNoteTag}
}
\SetTblrTemplate{note-tag}{normal}
\usepackage{longtable}
\usepackage{tabularx}
\usepackage{tabu}
\usepackage{lscape}
\usepackage{colortbl}  
\usepackage{booktabs}
\usepackage{multirow}
\usepackage{makecell}

