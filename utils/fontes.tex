%%%%%%%%%%%%%%%%%%%%%%%%%%%%%%%%%%%%%%%%%%%%%%%%%%%%%%%%%%%%%%%%%%%%%%%%%%%%%%%%
% Por: Abrantes Araújo Silva Filho
%      abrantesasf@gmail.com
% URL: https://github.com/abrantesasf/latex


%%%%%%%%%%%%%%%%%%%%%%%%%%%%%%%%%%%%%%%%%%%%%%%%%%%%%%%%%%%%%%%%%%%%%%%%%%%%%%%%
%%% Configurações de encodings e fontes:
\ifxetex
   % Se usar XeLaTeX, usa fontes específicas:
   \usepackage[tuenc,no-math]{fontspec}
   \setmainfont{equity-text-a-regular.otf}[
      Path           = /home/abrantesasf/.local/share/fonts/ ,
      BoldFont       = equity-text-a-bold.otf                ,
      ItalicFont     = equity-text-a-italic.otf              ,
      BoldItalicFont = equity-text-a-bold-italic.otf]
   \setmonofont{triplicate-t4-regular.otf}[
      Path           = /home/abrantesasf/.local/share/fonts/ ,
      BoldFont       = triplicate-t4-bold.otf                ,
      ItalicFont     = triplicate-t4-italic.otf              ,
      BoldItalicFont = triplicate-t4-bold-italic.otf]
\else
   % Se não for XeLaTeX, vai com as normais mesmo:
   \usepackage[T1]{fontenc}
   \usepackage[utf8]{inputenc}
   % Altera a fonte padrão do documento (nem todas funcionam em modo math):
   %   phv = Helvetica
   %   ptm = Times
   %   ppl = Palatino
   %   pbk = bookman
   %   pag = AdobeAvantGarde
   %   pnc = Adobe NewCenturySchoolBook
   \renewcommand{\familydefault}{ppl}
\fi
