%%%%%%%%%%%%%%%%%%%%%%%%%%%%%%%%%%%%%%%%%%%%%%%%%%%%%%%%%%%%%%%%%%%%%%%%%%%%%%%%
% fontes.tex
%
% Arquivo de configuração de packages para o LaTeX, considerando diversas
% classes que constumo utilizar.
%
% Para maiores informações, visite:
%    https://github.com/abrantesasf/latex
%    https://github.com/uvv-computacao/uvvtex2
%
% Você deve informar abaixo quais são as fontes que você quer usar, se utilizar
% o XeLaTeX para a compilação. Se compilar com o pdfLaTeX, usaremos as fontes
% padronizadas do TeX. Se usar fontes específicas, elas devem estar disponíveis
% no diretório "utils/fontes".
%
% NÃO ALTERE SE NÃO SOUBER O QUE ESTÁ FAZENDO!
%%%%%%%%%%%%%%%%%%%%%%%%%%%%%%%%%%%%%%%%%%%%%%%%%%%%%%%%%%%%%%%%%%%%%%%%%%%%%%%%


%%%%%%%%%%%%%%%%%%%%%%%%%%%%%%%%%%%%%%%%%%%%%%%%%%%%%%%%%%%%%%%%%%%%%%%%%%%%%%%%
%%% Configurações de encodings e fontes:
\ifxetex
  % Se usar XeLaTeX, usa fontes específicas:
  \usepackage{fontspec}
  \setmainfont[
    Path           = utils/fontes/                         ,
    UprightFont    = equity-text-a-regular.otf             ,
    BoldFont       = equity-text-a-bold.otf                ,
    ItalicFont     = equity-text-a-italic.otf              ,
    BoldItalicFont = equity-text-a-bold-italic.otf         ,
    SmallCapsFont  = equity-caps-a-regular.otf]{Equity Text A}
  \setsansfont[
    Path           = utils/fontes/                         ,
    UprightFont    = concourse_4_regular.otf               ,
    BoldFont       = concourse_4_bold.otf                  ,
    ItalicFont     = concourse_4_italic.otf                ,
    BoldItalicFont = concourse_4_bold_italic.otf           ,
    SmallCapsFont  = concourse_4_caps_regular.otf]{Concourse 4}
  \setmonofont[
    Path           = utils/fontes/                         ,
    UprightFont    = courier_prime.ttf                     ,
    BoldFont       = courier_prime_bold.ttf                ,
    ItalicFont     = courier_prime_italic.ttf              ,
    BoldItalicFont = courier_prime_bold_italic.ttf]{Courier Prime}
  \usepackage{fontawesome5}
\else
  % Se não for XeLaTeX, vai com as normais mesmo:
  \usepackage[T1]{fontenc}
  \usepackage[utf8]{inputenc}
  \usepackage{lmodern}
  % Altera a fonte padrão do documento (nem todas funcionam em modo math):
  %   phv = Helvetica
  %   ptm = Times
  %   ppl = Palatino
  %   pbk = bookman
  %   pag = AdobeAvantGarde
  %   pnc = Adobe NewCenturySchoolBook
  %\renewcommand{\familydefault}{ppl}
\fi
