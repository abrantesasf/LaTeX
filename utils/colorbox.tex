%%%%%%%%%%%%%%%%%%%%%%%%%%%%%%%%%%%%%%%%%%%%%%%%%%%%%%%%%%%%%%%%%%%%%%%%%%%%%%%%
% colorbox.tex
%
% Arquivo de configuração de packages para uso o LaTeX, conforme minhas
% preferências e modelos pessoais.
%
% Para maiores informações, visite:
%    https://github.com/abrantesasf/latex
%
% NÃO ALTERE SE NÃO SOUBER O QUE ESTÁ FAZENDO!
%%%%%%%%%%%%%%%%%%%%%%%%%%%%%%%%%%%%%%%%%%%%%%%%%%%%%%%%%%%%%%%%%%%%%%%%%%%%%%%%


%%%%%%%%%%%%%%%%%%%%%%%%%%%%%%%%%%%%%%%%%%%%%%%%%%%%%%%%%%%%%%%%%%%%%%%%%%%%%%%%
%%% Ativa suporte para boxes coloridos
\usepackage[many]{tcolorbox}

% Caixa vermelha
\newcommand{\boxvermelho}[2]{%
\begin{tcolorbox}[colback=red!5!white,colframe=red!75!black,
       breakable,enhanced,before upper={\parindent15pt\noindent},
       title=\textbf{#1}]
#2
\end{tcolorbox}}

% Caixa azul
\newcommand{\boxazul}[2]{%
\begin{tcolorbox}[colback=blue!5!white,colframe=blue!75!black,
       breakable,enhanced,before upper={\parindent15pt\noindent},
       title=\textbf{#1}]
#2
\end{tcolorbox}}

% Caixa verde
\newcommand{\boxverde}[2]{%
\begin{tcolorbox}[colback=green!5!white,colframe=green!75!black,
       breakable,enhanced,before upper={\parindent15pt\noindent},
       title=\textbf{#1}]
#2
\end{tcolorbox}}

% Caixa amarelo
\newcommand{\boxamarelo}[2]{%
\begin{tcolorbox}[colback=yellow!5!white,colframe=yellow!75!black,
       breakable,enhanced,before upper={\parindent15pt\noindent},
       title=\textbf{#1}]
#2
\end{tcolorbox}}

% Faixa azul
\newcommand{\faixaazul}[1]{%
\begin{tcolorbox}[top=0pt,bottom=0pt,left=0mm,right=0mm,colback=blue!75!black,
       colframe=blue!75!black,coltext=white]
\textbf{#1}
\end{tcolorbox}}

% Faixa cinza
\newcommand{\faixacinza}[1]{\vspace{0.3cm}%
\begin{tcolorbox}[top=0pt,bottom=0pt,left=0mm,right=0mm,
       colback=Gray!15!white,colframe=Gray!15!white,coltext=black]
\textbf{#1}
\end{tcolorbox}\vspace{-0.2cm}}

% Box de codigo
\newcommand{\boxcodigo}[0]{%
\protect\begin{tcolorbox}
\begin{verbatim}
x
\end{verbatim}
\end{tcolorbox}
}
