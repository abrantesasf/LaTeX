%%%%%%%%%%%%%%%%%%%%%%%%%%%%%%%%%%%%%%%%%%%%%%%%%%%%%%%%%%%%%%%%%%%%%%%%%%%%%%%%
% customizacoes.tex
%
% Arquivo de configuração de packages para uso o LaTeX, conforme minhas
% preferências e modelos pessoais.
%
% Para maiores informações, visite:
%    https://github.com/abrantesasf/latex
%
% NÃO ALTERE SE NÃO SOUBER O QUE ESTÁ FAZENDO!
%%%%%%%%%%%%%%%%%%%%%%%%%%%%%%%%%%%%%%%%%%%%%%%%%%%%%%%%%%%%%%%%%%%%%%%%%%%%%%%%


%%%%%%%%%%%%%%%%%%%%%%%%%%%%%%%%%%%%%%%%%%%%%%%%%%%%%%%%%%%%%%%%%%%%%%%%%%%%%%%%
%%% Vários comandinhos úteis e outras gambiarras

% Commando para ``italizar´´ palavras em inglês (e outras línguas!):
\newcommand{\ingles}[1]{\textit{#1}}

% Utilitários para a classe Exam:
\makeatletter
\@ifclassloaded{exam}{%
   % Linhas para respostas
   \setlength\linefillthickness{0.2pt}
   \newcommand{\umalinha}{\fillwithlines{0.25in}}
   \newcommand{\duaslinhas}{\fillwithlines{0.50in}}
   \newcommand{\treslinhas}{\fillwithlines{0.75in}}
   \newcommand{\quatrolinhas}{\fillwithlines{1.00in}}
   \newcommand{\cincolinhas}{\fillwithlines{1.25in}}
   \newcommand{\seislinhas}{\fillwithlines{1.50in}}
   \newcommand{\setelinhas}{\fillwithlines{1.75in}}
   \newcommand{\oitolinhas}{\fillwithlines{2.00in}}
   \newcommand{\novelinhas}{\fillwithlines{2.25in}}
   \newcommand{\dezlinhas}{\fillwithlines{2.50in}}
   
   % Verdadeiro ou Falso
   \newcommand{\vf}[1][{}]{%
      \fillin[#1][0.25in]%
   }
   
   % Título das respostas:
   %\renewcommand{\solutiontitle}{\noindent\textbf{Resposta:}\par\noindent}
   \renewcommand{\solutiontitle}{\noindent}
   %\shadedsolutions
   %\SolutionEmphasis{\itshape}
   
   % Altera a marca de questão correta e a ênfase:
   \checkedchar{$\Rightarrow$}
   %\CorrectChoiceEmphasis{\color{red}}
}{}
\makeatother

% Comando para escrever uma função e símbolos em fonte monoespaçada:
\newcommand{\funcao}[1]{\textbf{\texttt{#1}}}
\newcommand{\simbolo}[1]{\texttt{#1}}

% Commando para colocar o espaço correto entre um número e sua unidade:
%\newcommand{\unidade}[2]{\ensuremath{#1\,\mathrm{#2}}}
%\newcommand{\unidado}[2]{{#1}\,{#2}}

% Produz ordinal masculino ou feminino dependendo do segundo argumento:
%\newcommand{\ordinal}[2]{%
%#1%
%\ifthenelse{\equal{a}{#2}}%
%{\textordfeminine}%
%{\textordmasculine}}

% Comando para colocar o autor da citação corretamente justificado à direita:
% Modificado de: https://latex.org/forum/viewtopic.php?p=15605&sid=216895679326b531e85caec779e1d710#p15605
%\newcommand*{\fontequote}[2]{%
%   \unskip\hspace*{1em plus 1fill}%
%   \linebreak\hspace*{\fill}\mbox{#1}\vspace{-0.15cm}
%   \linebreak\hspace*{\fill}\mbox{#2}
%}
