%%%%%%%%%%%%%%%%%%%%%%%%%%%%%%%%%%%%%%%%%%%%%%%%%%%%%%%%%%%%%%%%%%%%%%%%%%%%%%%%
% customizacoes.tex
%
% Arquivo de configuração de packages para uso o LaTeX, conforme minhas
% preferências e modelos pessoais.
%
% Para maiores informações, visite:
%    https://github.com/abrantesasf/latex
%
% NÃO ALTERE SE NÃO SOUBER O QUE ESTÁ FAZENDO!
%%%%%%%%%%%%%%%%%%%%%%%%%%%%%%%%%%%%%%%%%%%%%%%%%%%%%%%%%%%%%%%%%%%%%%%%%%%%%%%%


%%%%%%%%%%%%%%%%%%%%%%%%%%%%%%%%%%%%%%%%%%%%%%%%%%%%%%%%%%%%%%%%%%%%%%%%%%%%%%%%
%%% Vários comandinhos úteis e outras gambiarras

% Commando para ``italizar´´ palavras em inglês (e outras línguas!):
\newcommand{\ingles}[1]{\textit{#1}}

% Comando para escrever uma função e símbolos em fonte monoespaçada:
\newcommand{\funcao}[1]{\textbf{\texttt{#1}}}
\newcommand{\simbolo}[1]{\texttt{#1}}

% Commando para colocar o espaço correto entre um número e sua unidade:
\newcommand{\unidade}[2]{\ensuremath{#1\,\mathrm{#2}}}
\newcommand{\unidado}[2]{{#1}\,{#2}}

% Produz ordinal masculino ou feminino dependendo do segundo argumento:
\newcommand{\ordinal}[2]{%
#1%
\ifthenelse{\equal{a}{#2}}%
{\textordfeminine}%
{\textordmasculine}}

% Comando para colocar o autor da citação corretamente justificado à direita:
% Modificado de: https://latex.org/forum/viewtopic.php?p=15605&sid=216895679326b531e85caec779e1d710#p15605
%\newcommand*{\fontequote}[2]{%
%   \unskip\hspace*{1em plus 1fill}%
%   \linebreak\hspace*{\fill}\mbox{#1}\vspace{-0.15cm}
%   \linebreak\hspace*{\fill}\mbox{#2}
%}
